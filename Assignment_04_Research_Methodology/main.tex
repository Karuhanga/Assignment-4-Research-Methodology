\documentclass{article}
\usepackage[utf8]{inputenc}
\usepackage{dtklogos} % for BibTeX stylized logo

\title{A Product Review for the Firebase Product Catalog}
\author{Lincoln Karuhanga \\ 216003532 \\ 16/U/5465/PS}
\date{Mar 2018}

\begin{document}

\maketitle

\newpage

\section{Introduction}
Firebase is a mobile and web application development platform and back-end service developed by Firebase, Inc. in 2011, then acquired by Google in 2014\textsubscript{[1]}. It offers several key services to developers that would ordinarily take edges for each product to independently implement. Firebase provides a suite of services including;
\begin{itemize}
    \item Firebase Analytics- A cost-free tool that helps developers monitor product use and engagement
    \item Firebase Cloud Messaging- An instant messaging service that can quickly be implemented by a product
    \item Firebase Auth- A handy user authentication service
    \item Firebase Real time Database- A back-end database service that allows synchronization of application data across various clients
    \item Firebase Test lab for android- Given the large variety of android devices on the market, Firebase provides a straight forward way of testing product performance on a range of devices
    \item Firebase Crash Reporting- Provides developers with real time reports of application performance and possible failure.
\end{itemize}
Others include Adwords, Admob, Invites, Dynamic Links, App Indexing, Notifications, hosting and  storage. This product evaluation focuses on the previously mentioned ones because they are unique offerings by Firebase.

\section{Overall Product Satisfaction}
As previously mentioned, in many of its offerings Firebase is the only company in that space. All the features mentioned in their catalog work as promised in the documentation.
Published developer reviews show a general satisfaction with the product. Paul Bresin from Vocabify(vocabifyapp.com) specifically lauds Firebase’s beautiful authentication, affordable database solutions, amazing and intuitive user interface, and decent integration libraries\textsubscript{[2]}.

\section{Cost}
Firebase offers several of it’s services free of charge, including analytics, crash monitoring and none telephone authentication. Firebase’s paid products are widely considered ‘affordable’ and at the minimum cheaper than other providers. Feedback collected from experienced Firebase users on Quora (quora.com) generally lauded the pay as you go plan offered by Firebase which allows one to pay for only the resources they use in a given time avoiding cost overcharge\textsubscript{[3,4]}.

\section{Trust and Transparency}
Firebase’s transparency has come into question several times, the largest of which was it’s feud with IOT company, Home Automation. The company reported an unexpected cost rise of 7000\%, without warning due to a change in the billing criteria. Reputation is a difficult thing to build once lost and Firebase took a hit on this one\textsubscript{[5]}.

\section{Performance}
Firebase’s features are built on top of some of the best infrastructure in the world, owned and managed by Google. This gives the company access to a distributed cloud infrastructure that helps make it’s services faster and more efficient. Concerns have been raised in regards to it’s cloud data store service which utilizes a NoSQL structure, that makes it difficult to write completed queries and makes large data set querying slightly less efficient\textsubscript{[6]}.

\section{Documentation and Support}
Firebase documentation is both precise and regularly reviewed and update. In my experience using the product for the last 6 months, I have not found a single unexplained, non counter intuitive feature. As evidenced in the previously mentioned Home Automation feud, the customer support is just as excellent as the company got in touch with the mentioned company to figure out a compromise\textsubscript{[5]}.

\nocite{*}
\bibliographystyle{IEEEtran}
\bibliography{IEEEabrv,IEEEexample}

\end{document}
